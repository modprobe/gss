\documentclass[a4paper,11pt]{scrartcl}

\usepackage[shortlabels]{enumitem}
\usepackage{csquotes}
\usepackage[utf8]{inputenc}
\usepackage[T1]{fontenc}
\usepackage[top=1.3in, bottom=1in, left=1.0in, right=0.6in]{geometry}
\usepackage{fancyhdr}
\usepackage[hidelinks]{hyperref}
\usepackage{minted}

\author{Alexander Timmermann}
\title{GSS-Übungsblatt 1}
\date{}

\pagestyle{fancy}
\fancyhf{}
\fancyhead[L]{Alexander Timmermann}
\fancyhead[C]{GSS-Übungsblatt 1}
\fancyhead[R]{Seite~\thepage}

\usemintedstyle{tango}

\begin{document}

\maketitle
\thispagestyle{empty}

\section{Allgemeine Aussagen zur IT-Sicherheit}

\begin{enumerate}[1.]
    \item
        \begin{displayquote}
            ,,Ein verteiltes System ist eine Menge voneinander abhängiger Computer,
            die dem Benutzer wie ein einzelnes, kohärentes System erscheinen.''%
            \footnote{Tanenbaum, van Steen: Verteilte Systeme. Pearson, 2. Auflage, 2008}
        \end{displayquote}

        Beispiele sind unter Anderem das Internet, ein internes Netzwerk wie z.B.
        das des Informatik-RZ oder ein verteiltes soziales Netzwerk wie Diaspora.

    \item Durch den Charakter eines verteilten Systems entstehen natürlicherweise
          mehr Angriffspunkte als bei einem zentralisierten System, da viele Computer
          koexistieren und jeder davon ein Einfallstor sein kann. Weiterhin gibt
          es offene Übertragungswege, die als Angriffspunkt dienen können.
          In verteilten Systemen kann es oft auch mehrere Betreiber geben, unter
          denen es jedoch bei einem Angriff oft nur zu mangelhafter Kommunikation
          kommt.

          Bei verteilten Systemen ist aber nicht immer gegeben, dass man mit einem
          Angriff bzw. mit einem Eindringen in eine der Komponenten auch gleich
          Zugriff auf alle gewünschten Daten hat, da diese möglicherweise von
          einem anderen System verwaltet werden.

    \item
        \begin{enumerate}[(a)]
            \item \textbf{Mangelnde Kompetenz der Nutzer}:
                  Oft sind gerade Nutzer, die nicht technisch versiert sind,
                  nicht in der Lage, z.B. eine Phishing-Attacke als solche zu
                  erkennen. Dadurch kann das Unternehmen kompromittiert werden.
            \item \textbf{Komfort vs. Sicherheit}:
                  Sichere Software bietet (leider) selten eine komfortable
                  User Experience. Es ist jedoch mittlerweile auch in nicht-technischen
                  Bereichen unerlässlich, Software einzusetzen. Diese muss
                  dementsprechend intuitiv zu bedienen sein.
            \item \textbf{Kostenfaktor vs. Sicherheit}:
                  Industrielle Software, die strikte Sicherheitsstandards
                  einhält, ist oft teuer. Viele Unternehmen erkennen den Wert
                  solcher Software nicht an und greifen zu einer
                  kostengünstigeren und oft unsichereren Lösung.
        \end{enumerate}
\end{enumerate}

\section{Schutzziele}

\begin{enumerate}[1.]
    \item
        \begin{enumerate}[(a)]
            \item \textbf{Anonymität, Pseudonymität und Unbeobachtbarkeit}\\
                  Die Begriffe beschreiben Abstufungen der Verschleierung der
                  Identität eines Individuums und/oder seiner Handlungen.
                  Anonymität ist dabei die stärkste Ausprägung, da hier vom
                  Individuum nichts bekannt ist. Bei Pseudonymität ist wenigstens
                  ein von der Person gewähltes Pseudonym bekannt, z.B. ein
                  Username. Unbeobachtbarkeit hingegen bezeichnet nur den Schutz
                  der Aktivitäten vor dem Einblick Dritter, d.h. dass z.B. das
                  Mitlesen von Nachrichten ausgeschlossen wird. Ein Schutz der
                  Identität ist hierbei nicht notwendigerweise eingeschlossen.
            \item \textbf{Vertraulichkeit und Verdecktheit}\\
                  Diese Begriffe umfassen die Inhalte von Kommunikationsdaten.
                  Bei Vertraulichkeit wird dieser Inhalt geschützt und kann nur
                  von den Beteiligten eingesehen werden. Bei der Verdecktheit
                  werden die Inhalte zusätzlich so übertragen, dass von Dritten
                  nicht einmal die Existenz von vertraulichen Inhalten erkannt
                  werden kann.
        \end{enumerate}

    \item
        \begin{enumerate}[(a)]
            \item \textbf{Integrität und Zurechenbarkeit}\\
                  Beide Begriffe umfassen das Senden und Empfangen von Daten.
                  Während das Schutzziel der Integrität jedoch gewährleistet,
                  dass Daten nach dem Senden weder vom Sender, noch von Dritten
                  unbemerkt geändert wurden, weist die Zurechenbarkeit nur nach,
                  ob bzw. dass Daten gesendet und empfangen wurden. Als Gegenteil
                  dazu kann die \textit{plausible deniability} (glaubwürdige
                  Abstreitbarkeit) angesehen werden.
            \item \textbf{Verfügbarkeit und Erreichbarkeit}\\
                  Beide Begriffe beschreiben den ordnungsgemäßen Betrieb von
                  verteilten Systemen. Während Verfügbarkeit jedoch die Bereitschaft
                  zur Benutzung des Systems beinhaltet, bezeichnet die
                  Erreichbarkeit nur, dass alle Komponenten ansprechbar sind,
                  nicht ob sie auch korrekt zusammenarbeiten.
        \end{enumerate}

    \item
        \begin{description}
            \item[Anonymität:]
                Als Betreiber einer Plattform kann man Anonymität gewährleisten,
                indem man keine Identifizierung fordert und auch keine Daten
                aufzeichnet, die zur Identifizierung genutzt werden könnten,
                z.B. Geodaten aus der IP-Adresse. Als Nutzer kann man sich hier
                nur teilweise mittels Software wie Tor oder VPNs schützen.
            \item[Pseudonymität:]
                Pseudonymität ist gewährleistet, wenn man als Betreiber keine
                Nutzer mit ihrer Identität identifiziert, sondern mit einem
                Pseudonym. Persönliche Daten dürfen dazu nicht erfasst werden.
            \item[Unbeobachtbarkeit:]
                Unbeobachtbarkeit ist gegeben, wenn man auch als Betreiber
                keine Daten mitlesen kann. Dazu können Daten beispielsweise
                clientseitig verschlüsselt werden.
            \item[Vertraulichkeit:]
                Vertraulichkeit kann z.B. mittels einer Verschlüsselung mit
                Public-Private-Key-Architektur umgesetzt werden. Am Beispiel
                eines Chat-Anbieters kann dann schon im Client verschlüsselt
                werden, die Übertragung geschieht dann geschützt vor dem
                Einblick Dritter.
            \item[Verdecktheit:]
                Zur Verdecktheit müssen die Daten, die ausgetauscht werden sollen,
                so versteckt werden, dass sie im Normalfall nicht als vertrauliche
                Daten erkannt werden. Hierzu kann man sich beispielsweise der
                Steganografie bedienen und die Daten in einem Bild verstecken.
                Dies ist jedoch höchst selten praktikabel.
            \item[Integrität:]
                Zur Sicherstellung der Integrität kann man beispielsweise
                Checksummen verwenden. Dazu wird ein Checksummen-Algorithmus
                benutzt und die resultierende Checksumme mit den Daten versandt.
                Wird nun etwas an den Daten verändert, ändert sich auch die
                Checksumme.
            \item[Zurechenbarkeit:]
                Durch serverseitiges Loggen kann nachgewiesen werden, wann und
                ob Daten versandt wurden. Natürlich funktioniert dies nur so lange
                es überhaupt einen Server gibt (also kein P2P-Netzwerk) und der
                Server nicht kompromittiert ist.
            \item[Verfügbarkeit:]
                Die Verfügbarkeit eines Systems kann immer nur nach bestem Wissen
                und bei regelmäßiger Überprüfung gewährleistet werden. Zum Zwecke
                der Überprüfung kann man jedoch eine Monitoring-Software
                einsetzen, die bei einem Ausfall alarmiert und ggf. bereits
                Schritte zur Behebung einleitet. Zur Sicherstellung eines
                möglichst ausfallfreien Betriebs kann man das System verteilen
                und in mehreren Aspekten redundant aufbauen (geografisch,
                funktional,\textellipsis).
            \item[Erreichbarkeit:]
                Ebenso wie Verfügbarkeit kann Erreichbarkeit nie garantiert
                werden. Durch Überwachungssysteme kann man einen Ausfall jedoch
                meist schnell feststellen und beheben. Zur Sicherstellung eines
                möglichst ausfallfreien Betriebs kann man das System verteilen
                und in mehreren Aspekten redundant aufbauen (geografisch,
                funktional,\textellipsis).
        \end{description}
\end{enumerate}

\section{Angreifermodell}
\label{sec:Angreifermodell}

\begin{enumerate}[1.]
    \item
        Als Angreifermodell bezeichnet man ein Modell, über das man die
        Stärke und Wirksamkeit eines Schutzmechanismus definieren kann.

        \begin{displayquote}
            ,,Das Angreifermodell definiert die maximal berücksichtigte Stärke eines
            Angreifers, gegen den ein Schutzmechanismus gerade noch wirkt.''%
            \footnote{Folien zur 1. Vorlesung, Seite 26}
        \end{displayquote}

        Es kann benutzt werden, um einen existierenden Schutzmechanismus
        einzuordnen, und gibt ggf. auch Hinweise auf
        Verbesserungsmöglichkeiten.

        Der Angreifende wird hierbei in verschiedenen Kategorien modelliert:

        \begin{description}
            \item[Rolle:]
                Wie viel Informationen über und Zugriff auf das System steht dem
                Angreifenden zur Verfügung? Bspw. kann davon ausgegangen werden,
                dass ein Wartungsdienst wesentlich mehr Informationen und
                Zugriff besitzt als ein normaler Benutzer (vgl. Insider $\leftrightarrow$ Outsider).
            \item[Verbreitung:]
                An welchen Stellen kann der Angreifende Informationen abgreifen
                oder verändern? Im kleinsten Fall hat ein Agitator nur
                eingeschränkte Benutzerrechte und kann nur wenige Informationen
                lesen oder verändern. In einem unwesentlich schlimmeren Fall
                hat ein Angreifer beispielsweise Zugriff auf allen Traffic an
                einem Internet Exchange Point (IXP) eines Landes (in Deutschland
                wäre das das DE-CIX in Frankfurt).
            \item[Verhalten:]
                Verhält sich ein Angreifer aktiv oder passiv? Sammelt er nur
                Daten, bspw. durch Mitschneiden von Netzwerktraffic, oder werden
                aktiv Daten zerstört?
            \item[Rechenkapazität:]
                Besonders bei kryptographischen Schutzmechanismus ist bedeutend,
                über wie viel Rechenkapazität ein Angreifer verfügt. Diese
                Schutzmechanismen beruhen auf Problemen, die sich mit ,,normaler'',
                d.h. kommerziell erschwinglich verfügbarer Hardware,  nicht
                lösen lassen. Verfügt ein Angreifer jedoch über eine große
                Rechenkapazität ist ein Schutz nicht notwendigerweise gewährleistet.
                Besonders wenn der Angreifer ein Staat bzw. eine Behörde ist
                (wie z.B. die \textit{National Security Agency} der USA oder der
                Bundesnachrichtendienst) kann von einer enormen Rechenleistung
                ausgegangen werden.
         \end{description}

    \item TODO
\end{enumerate}

\section{Angriffsformen}
\label{sec:Angriffsformen}

\begin{enumerate}[1.]
    \item
        Wenn wir einem der Lieferdienste Spionageabsichten unterstellen, könnte
        das Schutzziel der Vertraulichkeit kompromittiert werden. Beispielsweise
        könnte ein feindlicher Agent einen Datenträger mit einem Trojaner
        einschleusen. Sollten dann auch Sabotageabsichten hinzukommen, könnte
        z.B. auch Schadcode eingeschleust werden, der Daten zerstört und damit
        den ordnungsgemäßen Betrieb bzw. das Schutzziel der Verfügbarkeit stört.
        Werden von diesem Schadcode zum Zwecke der Sabotage Daten aktiv verändert
        ist auch das Schutzziel der Integrität betroffen.

    \item
        Auch hier ist eine Brechung aller drei Schutzziele denkbar. Bei einem
        normalen WPA-Netzwerk gibt es keinerlei Authentifizierung der APs,
        man kann also (solange man das Passwort des ursprünglichen Netzes kennt)
        einfach ein Netzwerk mit der gleichen SSID und dem gleichen Passwort
        erstellen, und solange der eigene AP der leistungsstärkere ist, werden
        sich alle Clients in Reichweite verbinden. Danach kann man Traffic
        mitschneiden (Vertraulichkeit), teilweise manipulieren (Integrität)
        oder auch bestimmten Traffic blockieren (Verfügbarkeit).
\end{enumerate}

\section{Passwortsicherheit}
\label{sec:Passwortsicherheit}

\begin{enumerate}[1.]
    \item TODO
    \item
        Wenn Passwörter im Klartext gespeichert werden, sind sie für alle
        ersichtlich, die Zugriff auf diese Datenbank haben, sei dieser Zugriff
        legitim oder illegitim. Bei einem massiven Leak würden Passwörter
        dabei ungeschützt verloren gehen. Über die Hälfte aller Internetnutzer
        benutzt weniger als 5 Passwörter in ihrem gesamten Online-Leben%
        \footnote{TeleSign Consumer Account Security Report, 2015,\\%
        \url{https://www.telesign.com/wp-content/uploads/2015/06/%
        TeleSign-Consumer-Account-Security-Report-2015-FINAL.pdf}},
        weshalb mit hoher Wahrscheinlichkeit auch weitere Accounts bei unbeteiligten
        Services kompromittiert würden.

        Bei Benutzung einer kryptografischen Hashfunktion wird eine Hashfunktion
        benutzt, die sich nicht umkehren lässt, d.h. aus dem berechneten
        Hashwert lässt sich der Ausgangstext nicht wiederherstellen. Benutzt man
        dieses System zur Kennwortspeicherung, so wird beim Setzen des Passworts
        der Hashwert berechnet und in einer Datenbank gespeichert. Versucht der
        Benutzer sich nun mit einem Passwort anzumelden, wird davon ebenfalls
        der Hashwert berechnet. Sind beide Hashwerte gleich hat der User das
        korrekte Passwort eingegeben.

    \item TODO

    \item
        Durch das Hinzufügen eines Salts wird der Hashwert in unvorhersehbarer
        Weise verändert. Man müsste also die Rainbow-Table für einen spezifischen
        Salt neu berechnen. Wird nun für jedes Passwort ein zufälliger Salt
        gewählt, wird die Rainbow-Table praktisch nutzlos.

    \item
        Als Wortliste wird das deutsche Wörterbuch von \textit{GNU Aspell}
        benutzt, das (ggf. nach Installation) unter
        \texttt{/usr/share/dict/ngerman} zu finden ist. Das Programm wurde in
        Ruby geschrieben und ist sehr kurz:

        \inputminted[linenos,numbersep=12pt,autogobble,frame=lines,framesep=2mm]%
                    {ruby}{pwcrack.rb}

        Durch die Kommentare im Code sollte die Funktionsweise klar sein. Führen
        wir es aus, so erhalten wir als Ergebnis:

        \texttt{Password is ``sonne''}\\
        \texttt{Took 0.102220636 seconds}

        Das Programm hat das Passwort also schon nach etwa $0,1$s geknackt. Wenn
        wir die Zeit erst nach Einlesen Wörterbuchs starten, so dauert es sogar
        nur $0,009$s.

        Wenn uns der Salt nicht bekannt wäre, so müssten wir für jedes Wort auch
        jeden Salt ausprobieren. Die Komplexität steigt damit stark an. Wenn
        zusätzlich auch keine Constraints vom Salt bekannt sind (z.B. max. 12
        Stellen, alphanumerisch, usw.), ist es praktisch unmöglich, den Hash zu
        knacken.
\end{enumerate}

\end{document}
