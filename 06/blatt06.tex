\documentclass[a4paper,11pt,ngerman]{scrartcl}

\usepackage[shortlabels]{enumitem}
\usepackage[utf8]{inputenc}
\usepackage{csquotes}
\usepackage[T1]{fontenc}
\usepackage[top=1.3in, bottom=1in, left=1.0in, right=0.6in]{geometry}
\usepackage{fancyhdr}
\usepackage[hidelinks]{hyperref}
\usepackage{tikz}
\usepackage{array}
\usepackage{amsmath}
\usepackage{setspace}
\usepackage{lastpage}
\usepackage{graphicx}
\usepackage{minted}
%\usepackage{blockgraph}
\usepackage{icomma}

\author{Alexander Timmermann, Jannis Krämer}
\title{GSS-Übungsblatt 6}
\date{}

\pagestyle{fancy}
\fancyhf{}
\fancyhead[C]{Alexander Timmermann, Jannis Krämer}
\fancyhead[L]{Blatt 6}
\fancyhead[R]{Seite~\thepage\ von \pageref{LastPage}}

\begin{document}

\maketitle
\thispagestyle{empty}

\doublespace

\section*{Aufgabe 1: Zentrale Begriffe der Kryptographie}
  \begin{enumerate}[\bf 1.]
    \setcounter{enumi}{1}
    \item
      Bei einem symetrischen Kryptosystem wird für Ver- und Entschlüsselung
      nur ein Schlüssel verwendet, bei einem asymetrischen jeweils ein distinkter für
      Ver- und Entschlüsselung. Bei einem symetrischen Verfahren müssen demnach
      für jede mögliche Kombination aus sendender Person und empfangender Person ein
      Schlüssel generiert werden, insgesamt also $\binom{n}{2}$. Bei einem asymetrischen
      Kryptosystem benötigt jede Person einen öffentlichen und einen privaten Schlüssel
      und kann danach durch Herausgabe des öffentlichen Schlüssels und bei Geheimhaltung
      des privaten Schlüssels mit allen anderen Gruppenmitgliedern kommunizieren ohne
      dass jemand unbefugtes die Nachricht entschlüsseln könnte. Hierbei werden insgesamt
      also nur $2n$ Schlüssel benötigt.
    \item
      \begin{itemize}
        \item
          Aufgrund des hohen Rechenaufwandes ausymetrischer Kryptosysteme wird
          Alice für zeitkritische Übertragungen (Echtzeitapplikationen o.ä.) sowie
          für große Datenmengen. Auch für das Versenden einer Nachricht an mehrere
          Empfänger*innen ohne erneute Verschlüsselung kann Alice sich für ein
          hybrides Kryptosystem entscheiden.
        \item
          Alice erzeugt mit einem symmetrischen Verschlüsselungssystem einen
          neuen Schlüssel. Damit verschlüsselt sie die Nachricht. Den symmetrischen
          Schlüssel verschlüsselt sie ebenfalls, jedoch mit Bobs öffentlichen Schlüssel.
          Dann schickt sie beide Chiffres an Bob. Dieser kann nun zuerst den symetrischen
          Schlüssel entschlüsseln und mit diesem dann die Nachricht selbst entschlüsseln.
        \item
          Die Nachricht würde vorraussichtlich aus einem Header bestehen, in dem der
          symetrische Schlüssel übertragen wird und einem Teil für die codierte Nachricht.
          Die genaue Umsetzung ist dabei von Protokoll zu Protokoll unterschiedlich.
      \end{itemize}
  \end{enumerate}

\section*{Aufgabe 2: Parkhaus}
  \begin{enumerate}[\bf 1.]
    \setcounter{enumi}{1}
    \item
      Der Barcode, der vom Kassenautomaten auf das Ticket gestempelt wird, scheint
      immer der gleiche zu sein und kann somit wohl nur eine binäre Bestätigung des
      Bezahlens sein. Dieser, sowie die menschenlesbaren Informationen unten links,
      könnte ein*e Angreifer*in also auf das Ticket drucken und damit versuchen, gratis
      zu parken.

      Auch die Rabattbarcodes scheinen sich nicht sehr zu verändern und ließen sich
      nach Probieren verschiedener Kombinationen zu fälschen; in unser Stichprobe
      erscheinen nur drei unterschiedliche Rabattbarcodes.

      Dem System liegt offenbar ein Angreifermodell zugrunde dass von einem externen
      Angriff ausgeht, der zwar über beliebig viel Rechenkapazität, jedoch nicht
      über einen Barcodedrucker verfügen kann.
    \item
      Die besten Ergebnisse würde eine Umstellung des Ticket-Systems auf ein generell
      sicheres System (z.B. RFID-Parktickets) erzielen.

      Das vorhandene System würde davon profitieren, wenn sowohl in den Rabattcodes
      als auch im Kassenautomaten-Barcode das Datum und die Uhrzeit codiert wäre.
      Das würde das System zwar auch nicht endgültig absichern, jedoch wäre eine
      Entschlüsselung deutlich aufwendiger. Dies scheint bisher nicht der Fall zu sein.
  \end{enumerate}

\section*{Aufgabe 3: Authentifizierungsprotokolle}
  \begin{enumerate}[\bf 1.]
    \setcounter{enumi}{1}
    \item
      Ein*e Angreifer*in kann sich bei diesem System immer noch beim Server athentifizieren,
      indem sie*er als passive*r Angreifer*in während der Übertragung $r$ und $c$ abgreift.
      Da bei diesem Ansatz die Daten immer noch unverschlüsselt übertragen werden, ist
      dies ungehindert möglich.

      Einzig ein Angriff bei dem ein*e Angreifer*in versucht, die Login-Daten $u$ und $p$
      abzugreifen wird dadurch verhindert.
    \item
      Durch das Challenge-Response-System müsste ein*e Angreifer*in die Verschlüsselungsfunktion
      $E$ und den Schlüssel $k$ kennen um dem Server die korrekte Response zu liefern,
      ohne die es nun nicht mehr möglich ist sich als jemand anders auszugeben.

      Dieses System schützt jedoch nicht vor einem Man-in-the-Middle-Angriff, bei dem
      sich der*die Angreifer*in zwischen Server und Client schaltet und sowohl die
      Challenge des Servers als auch die Response des Client abgreift und diese dann
      weiterleitet, als hätte die Generierung auf dem angreifenden System stattgefunden.

      Nach der Authentifizierung kann durch die Man-in-the-Middle-Attack außerdem der
      (immer noch unverschlüsselte) Daten mitlesen und ggf. anpassen.
  \end{enumerate}

\section*{Aufgabe 5: RSA-Verfahren}
  Nach der Berechnung von $d=4343$ ergibt sich folgender entschlüsselter Text:
    \begin{quote}
      Fuer die GSS-Klausur sind folgende Themen wichtig: Angreifermodelle, Schutzziele,
      Rainbow Tables, die (Un-)Sicherheit von Passwoertern und dazugehoerige Angriffe,
      Zugangs- und Zugriffskontrolle, Timing-Attack und Power-Analysis, Biometrische Verfahren,
      Grundlagen der Kryptographie, das RSA-Verfahren, Authentifikationsprotokolle und
      natuerlich alle anderen Inhalte, die wir in der Uebung und der Vorlesung behandelt
      haben :-)
    \end{quote}
  Nachfolgend der benutzte Code:
  \inputminted[linenos,numbersep=12pt,autogobble,frame=lines,framesep=2mm]{python3}{rsadecrypt.py}
\end{document}
