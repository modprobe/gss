\documentclass[a4paper,11pt]{scrartcl}

\usepackage[shortlabels]{enumitem}
\usepackage{csquotes}
\usepackage[utf8]{inputenc}
\usepackage[T1]{fontenc}
\usepackage[top=1.3in, bottom=1in, left=1.0in, right=0.6in]{geometry}
\usepackage{fancyhdr}
\usepackage[hidelinks]{hyperref}
\usepackage{tikz}
\usepackage{array}
\usepackage{amsmath}
\usepackage{setspace}
\usepackage{lastpage}
\usepackage{blockgraph}

\author{Alexander Timmermann, Jannis Krämer}
\title{GSS-Übungsblatt 2}
\date{}

\pagestyle{fancy}
\fancyhf{}
\fancyhead[C]{Alexander Timmermann, Jannis Krämer}
\fancyhead[L]{Blatt 3}
\fancyhead[R]{Seite~\thepage\ von \pageref{LastPage}}

\begin{document}

\maketitle
\thispagestyle{empty}

\doublespace

\section{Scheduling-Algorithmen}

\begin{enumerate}[\bf a)]
    \item
\end{enumerate}

\section{Echtzeit- und Multiprozessor-Scheduling}

\begin{enumerate}[\bf a)]
    \item
\end{enumerate}

\section{Prioritätsinversion}

\begin{enumerate}[\bf a)]
    \item Es ergibt sich folgende Abbildung:
    \begin{center}
        \scalebox{0.4}{\begin{blockgraph}{34}{5}{1} % 170 / 5 = 34 cols
\bglabelxx{0}
\bglabelxx{5}
\bglabelxx{10}
\bglabelxx{15}
\bglabelxx{20}
\bglabelxx{25}
\bglabelxx{30}
\bglabelxx{34}

\bglabely{4}{Periodendauer B}
\bglabely{3}{Periodendauer Z}
\bglabely{2}{Periodendauer M}
\bglabely{1}{Berechnung}
\bglabely{0}{$Kontextswitch \star$}

\bgblock[1]{0}{2}{$B_1$}
\bgblock[1]{2}{6}{$Z_1$}
\bgblock[1]{6}{8}{$M_1 a$}
\bgblock[0]{8}{10}{$B_2$}
\bgblock[1]{10}{12}{$M_1 b$}
\bgblock[0]{12}{16}{$Z_2$}
\bgblock[0]{16}{18}{$B_3$}
\bgblock[1]{18}{20}{$M_1 c$}

\bgblock[2]{0}{23}{$M$}
\bgblock[2]{23}{34}{$M$} % 115 / 5 = 23
\bgblock[3]{0}{12}{$Z$}
\bgblock[3]{12}{24}{$Z$}
\bgblock[3]{24}{34}{$Z$} % 60 / 5 = 12
\bgblock[4]{0}{8}{$B$}
\bgblock[4]{8}{16}{$B$}
\bgblock[4]{16}{24}{$B$}
\bgblock[4]{24}{32}{$B$} % 40 / 5 = 8
\end{blockgraph}}
    \end{center}
    
    
    Während der Bearbeitung von M1 wird B aufgrund seiner höheren Priorität eingeschoben. M1 gibt dabei seine Mutexlocks weiter, obwohl M1 noch nicht alle Daten schreiben konnte. Im Zuge des Bus Management, das B ausführt, benötigt B nämlich auch Zugriff auf auf die Daten von M1. Nachdem B durchgelaufen ist läuft M1 somit weiter. Bevor M1 seinen Task jedoch beenden kann wird er abermals unterbrochen, diesmal jedoch von Z. Z läuft mit mittlerer Priorität, löst M1 somit ab. Z benötigt jedoch keinen Zugriff auf von M1 geschriebene Daten, erhält also auch nicht die Mutexlocks von M1. Nachdem Z nun fertig ist wird er allerdings von B abgelöst, nicht von M, da B die höchste Priorität besitzt. B1 hat nun keinen Zugriff auf Ms Daten, da M nicht aktiv ist und somit B auch keine Mutexlocks übertragen kann. B kann deshalb nicht zuende rechnen und muss warten bis M seine Daten fertig geschrieben hat. Da jedoch zuerst alle Prozesse mit höherer Priorität als M ausgeführt werden, kann die Ausführung B unter Umständen sehr lange verhindert sein und ein zeitkritisches System somit zum Absturz bringen. Der Computer der Pathfinder-Mission beispielweise führte automatisch einen Neustart, welcher mit Datenverlust verbunden war, aus wenn der Bus Management Task (B) zu lange nicht ausgeführt wurde. 
\end{enumerate}

\end{document}